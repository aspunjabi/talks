\documentclass{beamer}

\mode<presentation>
{
	\usetheme{Copenhagen}
	\usecolortheme{beaver}
	\setbeamercovered{transparent}
	\setbeamertemplate{items}[ball]
	\setbeamertemplate{theorems}[numbered]
	\setbeamertemplate{footline}[frame number]
}

\usepackage{beamerthemesplit}
\usepackage{graphics}
\usepackage{graphicx}
\usepackage{hyperref}
\usepackage{listings}

\title{ Data Analysis and Visualisation using R }
\author{ Vinayak Hegde }
\date{ July 11, 2013 }

\AtBeginSection[]  % "Beamer, do the following at the start of every section"
{
	\begin{frame}<beamer> 
	\frametitle{Outline} % make a frame titled "Outline"
	\tableofcontents[currentsection]  % show TOC and highlight current section
	\end{frame}
}
\begin{document}

\frame {
	\titlepage
}

\section*{Outline}
\frame {
	\frametitle{Outline of Topics}
	\tableofcontents
}

\section{ Introduction }
\frame {
	\frametitle { What is R ? }
	\begin{block}	{Wikipedia} R is a free software programming language and a software environment for statistical computing and graphics. The R language is widely used among statisticians and data miners for developing statistical software and data analysis.
	\end{block}
}

\frame {
	\frametitle { Why use R ? }
	\begin{itemize}
	\item Designed and optimised for data processing
		\item Lots of modules
		\item State of the art graphics
		\item Free as in freedom/beer
		\item Helpful community
		\item Very flexible and good integration
	\end{itemize}
}

\frame {
	\frametitle { R Studio Installation }
	\begin{itemize}
		\item Go to \href{http://www.rstudio.com/ide/download/}{{\alert {RStudio website}}}
		\item Download the server/desktop version
		\item For server - Open the browser and go to http://127.0.0.1:8787
		\item For desktop - Click on the shortcut and you are ready to go
	\end{itemize}
}

\frame {
	\frametitle { Basics - Starting off and getting help }
	\begin{itemize}
		\item Starting the interpreter
		\item Getting online help  - ? or help()
		\item Searching for help - ?? 
		\item Approximate search - apropos()
	\end{itemize}
}

\frame {
	\frametitle { Basics - Library commands }
	\begin{itemize}
		\item install.packages("packagename")
		\item library("package")
		\item update.packages("packages")
		\item search()
		\item detach("package:packagename")
	\end{itemize}	
}

\frame {
	\frametitle { Basics - Objects and Workspaces }
	\begin{itemize}
	\item attach(object) 
		\item detach(object)
		\item rm()
		\item save.image("ExploreData.RData")
		\item load("SavedWorkspace.RData")
		\item save(data1,data2,file="SavedWorkspace.RData")
	\end{itemize}
}

\frame {
	\frametitle { Basics - Using inbuilt function }
	\begin{itemize}
		\item str
		\item summary
		\item head
		\item View
		\item Assignment \textless-
	\end{itemize}
}

\frame {
	\frametitle { Basics - Inbuilt Statistics functions }
	\begin{itemize}
		\item mean
		\item sd
		\item var
		\item median
		\item quantile
		\item hist
		\item plot
	\end{itemize}
}

\section { Data Structures }
\frame {
	\frametitle { vector }
} 

\frame {
	\frametitle { Array }
} 

\frame {
	\frametitle { Matrix }
} 

\frame {
	\frametitle { List }
} 

\frame {
	\frametitle { Data Frames }
} 

\frame {
	\frametitle { Working with Data Structures }
	\begin{itemize}
		\item concatenate c()
		\item cbind()
		\item rbind()
		\item data.frame()
		\item mode()
		\item class()
	\end{itemize}
}

\frame {
	\frametitle { Working with Data Structures }
	\begin{itemize}
		\item with()
		\item sort()
		\item subset()
		\item select()
		\item transform()
	\end{itemize}
}

\frame {
	\frametitle { Working with Data Structures }
	\begin{itemize}
		\item names()
		\item row.names()
		\item attributes()
	\end{itemize}
}

\section { Working with Data }
\frame {
	\frametitle { Reading data from multiple Sources }
	\begin{itemize}
		\item Excel files
		\item web pages
		\item csv
		\item databases
	\end{itemize}
}

\frame {
	\frametitle { Transforming data }
}

\section { Packages and Datasets }
\frame {
	\frametitle { Packages }
	\begin{itemize}
		\item plyr
		\item ggplot2
		\item reshape2
		\item lattice
	\end{itemize}
}

\frame {
	\frametitle { Datasets }
	\begin{itemize}
		\item iris
		\item mtcars
		\item diamonds
		\item HairEyeColor
	\end{itemize}
}

\section { Visualisation }
\frame {
	\frametitle { Viz Packages }
	\begin{itemize}
		\item boxplot 
		\item ggplot2
		\item plot
		\item lattice
	\end{itemize}
}

\section { Webapps }
\frame {
	\frametitle { Shiny }
}

\section { Integration with other Systems }
\frame {
	\frametitle { Packages }
	\begin{itemize}
		\item hadoop 
		\item c++
		\item javascript
	\end{itemize}
}

\end{document}
